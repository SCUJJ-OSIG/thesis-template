%-------------------------------------------------
% FileName: chapt-5.tex
% Author: Safin (zhaoqid@zsc.edu.cn)
% Version: 0.1
% Date: 2020-05-12
% Description: 第5章
% Others: 
% History: origin
%------------------------------------------------- 


% 断页
% \clearpage
\chapter{系统实现与测试}

\section{系统实现}
介绍主要功能模块的编程实现以及系统的部署方法。

\section{系统测试}
阐述系统的测试技术、测试过程和测试结果。


% table环境
% [H] 浮动优先级,当前位置,但尺寸过大的浮动体可能使得分页比较困难

% [htbp!] 浮动方式 请参考一份(不太)简短的 LATEX 2" 介绍,3.9节
% h 当前位置(代码所处的上下文)
% t 顶部
% b 底部
% p 单独成页
% ! 在决定位置时忽视限制
% 排版位置的选取与参数里符号的顺序无关, 
% LATEX 总是以 h-t-b-p 的优先级顺序决定浮动体位置。
% 也就是说 [!htp] 和 [ph!t] 没有区别。

完全手动完成的表格,如表\ref{tab:tab1}所示。 % 通过label引用表格
\begin{table}[H] % H浮动优先级,当前位置
        \zihao{5} % 字号5
    \centering  % 居中
    \caption{一个表格}  % 表格标题
    \label{tab:tab1}  % 用于在正文中引用的label
    % 字母的个数对应列数,| 代表分割线
    % l代表左对齐,c代表居中,r代表右对齐
    \begin{tabular}{|c|c|c|c|}   
        \hline  % 表格的横线 
        1 & 2 & 3 & 4 \\  % 表格中的内容,用&分开,\\表示下一行
        \hline 
        0.1 & 0.2 & 0.3 & 0.4 \\
        \hline
    \end{tabular}
\end{table}

以下编辑器(TexStudio)的表格向导生成的表格,如表\ref{tab:tab2}所示。% 通过label引用表格
\begin{table}[H] % H浮动优先级,当前位置
        \zihao{5} % 字号5
	\centering  % 居中
	\caption{诗词曲} % 表格标题  
    \label{tab:tab2}   % 用于在正文中引用的label
    % 字母的个数对应列数,| 代表分割线
    % l代表左对齐,c代表居中,r代表右对齐
    \begin{tabular}{|c|c|c|c|}
        \hline  % 表格的横线 
          & 唐诗 & 宋词 & 元曲 \\ 
        \hline 
        1 & 李白 & 苏轼 & 关汉卿 \\ 
        \hline 
        2 & 白居易 & 辛弃疾 & 马致远 \\ 
        \hline 
        3 & 杜甫 & 李清照 & 张可久 \\ 
        \hline 
        4 & 王维 & 陆游 & 张养浩 \\ 
        \hline 
        5 & 孟浩然 & 欧阳修 & 徐再思 \\ 
        \hline 
    \end{tabular}  
\end{table}

嵌套表格,如表\ref{tab:tab3}所示。% 通过label引用表格
\begin{table}[H] % H浮动优先级,当前位置
        \zihao{5} % 字号5
	\centering  % 居中
	\caption{嵌套表格} % 表格标题  
    \label{tab:tab3}  % 用于在正文中引用的label
    % 字母的个数对应列数,| 代表分割线
    % l代表左对齐,c代表居中,r代表右对齐
    \begin{tabular}{|c|c|c|}
        \hline  % 表格的横线 
        a & b & c \\ \hline
        a & \multicolumn{1}{@{}c@{}|}
        {\begin{tabular}
{c|c}
            e & f \\ \hline
            e & f \\
        \end{tabular}}
        & c \\ \hline
        a & b & c \\ \hline
    \end{tabular}
\end{table}

控制列宽的表格,如表\ref{tab:tab4},表\ref{tab:tab5}所示。% 通过label引用表格
\begin{table}[H] % H浮动优先级,当前位置
    \zihao{5} % 字号5
    \centering  % 居中
    \caption{控制列宽的表格} % 表格标题  
    \label{tab:tab4}  % 用于在正文中引用的label 
    \begin{tabularx}{30em}  % 总列宽 30em
    % 多个 X 列格式平均分配列宽
        {|*{4}{>{\centering\arraybackslash}X|}}
        \hline  % 表格的横线 
        A & B & C & D \\ \hline
        a & b & c & d \\ \hline
    \end{tabularx}
\end{table}

\begin{table}[H] % H浮动优先级,当前位置
    \zihao{5} % 字号5
    \centering  % 居中
    \caption{诗词曲}  % 表格标题 
    \label{tab:tab5}  % 用于在正文中引用的label 
    % 总列宽 30em
    \begin{tabularx}{30em} 
    % 多个 X 列格式平均分配列宽
        {|*{4}{>{\centering\arraybackslash}X|}}
        \hline  % 表格的横线 
          & 唐诗 & 宋词 & 元曲 \\ 
        \hline 
        1 & 李白 & 苏轼 & 关汉卿 \\ 
        \hline 
        2 & 白居易 & 辛弃疾 & 马致远 \\ 
        \hline 
        3 & 杜甫 & 李清照 & 张可久 \\ 
        \hline 
        4 & 王维 & 陆游 & 张养浩 \\ 
        \hline 
        5 & 孟浩然 & 欧阳修 & 徐再思 \\ 
        \hline 
    \end{tabularx}  
\end{table}

控制行距的表格,如表\ref{tab:tab6}所示。% 通过label引用表格
\begin{table}[H] % H浮动优先级,当前位置
    \zihao{5} % 字号5
    \centering  % 居中
    \caption{控制行距的表格}  % 表格标题 
    \label{tab:tab6}  % 用于在正文中引用的label 
    \renewcommand\arraystretch{2.8} % 行距控制
    % 字母的个数对应列数,| 代表分割线
    % l代表左对齐,c代表居中,r代表右对齐
    \begin{tabular} {|c|c|c|c|}
        \hline % 表格的横线 
        A & B & C & D \\ \hline
        a & b & c & d \\ \hline
    \end{tabular}
\end{table}

表格单行内容太长,直接换行,如表\ref{tab:tab7},表\ref{tab:tab8}所示。% 通过label引用表格
\begin{table}[H] % H浮动优先级,当前位置
    \zihao{5} % 字号5
    \centering  % 居中
    \caption{单行内容太长直接换行}  % 表格标题 
    \label{tab:tab7}   % 用于在正文中引用的label
    % 字母的个数对应列数,| 代表分割线
    % l代表左对齐,c代表居中,r代表右对齐
    \begin{tabular}{|c|c|c|c|} 
        \hline  % 表格的横线 
          & 唐诗 & 宋词 & 元曲 \\ 
        \hline 
        1 & 李白李白李白 & 苏轼苏轼苏轼苏轼 & 关汉卿关汉卿关汉卿关汉卿 \\
         & 李白李白李白李白 & 苏苏轼苏轼轼 & 关汉卿关汉卿关汉卿 \\
        \hline 
        2 & 白居易 & 辛弃疾 & 马致远 \\ 
        \hline 
    \end{tabular}  
\end{table}


\begin{table}[H] % H浮动优先级,当前位置
    \zihao{5} % 字号5
    \centering  % 居中
    \caption{控制列宽单行内容太长直接换行}  % 表格标题 
    \label{tab:tab8}   % 用于在正文中引用的label 
    \begin{tabularx}{30em} % 总列宽 30em
    % 多个 X 列格式平均分配列宽
        {|*{4}{>{\centering\arraybackslash}X|}}
        \hline  % 表格的横线 
          & 唐诗 & 宋词 & 元曲 \\ 
        \hline 
        1 & 李白李白李白 & 苏轼苏轼苏轼苏轼 & 关汉卿关汉卿关汉卿关汉卿 \\
         & 李白李白李白李白 & 苏苏轼苏轼轼 & 关汉卿关汉卿关汉卿 \\
        \hline 
        2 & 白居易 & 辛弃疾 & 马致远 \\ 
        \hline 
    \end{tabularx}  
\end{table}

  